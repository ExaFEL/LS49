\documentclass[12pt, letterpaper]{article}
\usepackage[english]{babel}
\usepackage{amsmath}
\usepackage{unicode-math}
\usepackage[margin=0.7in]{geometry}
\usepackage{graphicx}
%\usepackage{mathrsfs}
\usepackage{fontspec}
\setmainfont{Calibri}
% must be compiled with XeLaTeX, not pdfLaTeX

  %%%%%%%%%%%%
  % PREAMBLE %
  %%%%%%%%%%%%
\begin{document}
\selectlanguage{english}

\section*{LS49: Profile modeling for spatially resolved anomalous dispersion. }
  \par The Jungfrau-measured Bragg spots are radially elongated, and the goal is to use the
  spot profiles to deduce the $f\prime$ and $f\prime\prime$ parameters for individual
  metal centers in the protein.  Section 1 will establish the general methodology and apply it 
  to background modeling.  Then the spot profile will be brought in and scale factors will be fit.
  Finally, the anomalous dispersion curves will be modeled. 
\subsection*{1 Modeling the backgound with a Bayesian approach}

  \par In this initial section we will consider every Bragg spot on every image separately.  
  The program \textit{DIALS} has already been run, and 2D shoeboxes defined for each Bragg 
  spot.  Furthermore the program has already identified which pixels in the rectangular shoebox are
  ``foreground'' and thus correspond to the Bragg signal, and which are background.  Since our
  Bragg spots are unusually elongated, the \textit{DIALS} analysis is not necessarily accurate, 
  but for our first demonstration we will use the pixels as currently flagged without further
  question.  In the next section, we will no longer need to make a distinction between foreground 
  and background, as all pixels in the shoebox will be treated uniformly.
  
  \par We'll use a planar model for the background, as used by Michael Rossmann and 
  subsequently by Andrew Leslie.  The $i^{th}$ pixel in the shoebox will have coordinates 
  $x_i, y_i$, taking $x$ as slow and $y$ as fast, with both zero-based.  The model background will then
  be expressed as 
    \begin{equation}
    \lambda_{i} = ax_i + by_i + c 
    \text{,}
    \label{eqn:RossmannB}
  \end{equation}
  where $a, b,$ and $c$ are the model parameters to be fit.  Importantly, the model is expressed in
  units of photons per pixel.
 
 \par   We'll take a Bayesian approach.  The probability of the model given the data is equal to the
 probability of the data given the model, times the probability of the model:
 
    \begin{equation}
    P(\textrm{model|data}) = P(\textrm{data|model}) \times P(\textrm{model})
    \text{.}
    \label{eqn:Bayes}
  \end{equation}

  For this section we'll treat the probability of the model as identically 1.  Furthermore the probability 
  of the data is assumed to be independent for each pixel, therefore the collective probability is the
  product of individual pixel probabilities, and the model likelihood $L(a,b,c)$ expressed as 
      \begin{equation}
    P(\textrm{model|data}) = \prod_{i=1}^{N_p} {P_i(\textrm{data|model}) }
    \text{,}
    \label{eqn:Bayes2}
  \end{equation}
  
  where $N_p$ is the number pixels. Taking the usual approach, we will minimize the negative 
  log likelihood function,
      \begin{equation}
    L(a,b,c) = -\ln (\prod_{i=1}^{N_p} {P_i(\textrm{data|model}) }) =
    -\sum_{i=1}^{N_p} \ln {P_i(\textrm{data|model}) }
    \text{,}
    \label{eqn:Bayes2}
  \end{equation}
 
   to find optimize the values of $a, b,$ and $c$.
   \par The data probability will be taken as arising from shot noise.  If photons reach the pixel at
   an average rate of $\lambda_i$ (so says our model), then the probability of observing $k_i$ 
   photons in a particular interval is given by the Poissonian distribution (see wikipedia):
   
    \begin{equation}
   P(k_i) = \frac{\lambda_i^{k_i} e^{-\lambda_i}}{k!}
    \text{.}
    \label{eqn:Poisson}
  \end{equation}

\par Combine the two equations and expand,
   
    \begin{equation}
   L =  -\sum_{i=1}^{N_p} \ln ( \frac{\lambda_i^{k_i} e^{-\lambda_i}}{k!} )
  = \sum_{i=1}^{N_p} ( \lambda_i - k_i \ln \lambda_i + \ln k! )
    \text{.}
    \label{eqn:terms}
  \end{equation}
\par The last term involving $k!$ is independent of the model parameters, is therefore constant, 
and of no consequence for our parameter fitting, so we drop it:

    \begin{equation}
   L = \sum_{i=1}^{N_p} ( \lambda_i - k_i \ln \lambda_i )
    \text{.}
    \label{eqn:simple}
  \end{equation}
\par The derivatives with respect to the model parameters are remarkably simple:

  \begin{equation}
   \dfrac{\partial L} {\partial a} = 
  \sum_{i=1}^{N_p} x_i (1 - \frac{ k_i}{ \lambda_i } )
    \text{,}
    \label{eqn:grada}
  \end{equation}
  \begin{equation}
   \dfrac{\partial L} {\partial b} = 
  \sum_{i=1}^{N_p} y_i (1 - \frac{ k_i}{ \lambda_i } )
    \text{.}
    \label{eqn:gradb}
  \end{equation}
 \begin{equation}
   \dfrac{\partial L} {\partial c} = 
  \sum_{i=1}^{N_p}  (1 - \frac{ k_i}{ \lambda_i } )
    \text{.}
    \label{eqn:gradb}
  \end{equation}

  \subsection*{2 The formula of Einsle \textit{et al.}}

  \par Consider the wavelength-dependent atomic scattering factor for X-rays, $f(\lambda)$. We 
  express it as a complex sum of a wavelength-independent part $f^0$ and a 
  wavelength-dependent term: 
  
    \begin{equation}
    f (\lambda)=f^0 + \Delta{f}\prime(\lambda) + i\Delta{f''}(\lambda)
    \text{.}
    \label{eqn:part}
  \end{equation}
  
  The structure factor for a Miller index $\mathbf{h}$ is 
     \begin{equation}
    \mathbf{F_h} =\left( \begin{array}{c}
      A \\
      B
    \end{array}\right)
    \text{.}
    \label{eqn:vecF}
  \end{equation}

  $\mathbf{F}$ is treated as a vector in the complex plane with real and imaginary parts A and B respectively.  We use
  vector notation expressed in the complex basis (1 $i$). As usual the structure is a linear combination of its parts:
  
       \begin{equation}
    \mathbf{F_h} =\left( \begin{array}{c}
      A \\
      B
    \end{array}\right)_{\textrm{bulk solvent}} +
    \left( \begin{array}{c}
      A \\
      B
    \end{array}\right)_{\textrm{protein atoms}} + \textrm{metals, etc.}
    \text{}
    \label{eqn:parts}
  \end{equation}

  Now focus on the contribution of one metal atom, $m$.
      \begin{equation}
    \mathbf{F}_{\mathbf{h},m} = f_m (\lambda)\exp(2\pi{i}(\mathbf{r}_m\cdot\mathbf{h}))
    \text{,}
    \label{eqn:vecF}
  \end{equation}
  where $\mathbf{r}_m$ is the position vector of metal $m$ in unit cell-fractional coordinates. Expand this
  and change form,
        \begin{equation}
    \mathbf{F}_{\mathbf{h},m} = [f^0_m + \Delta{f_m}\prime(\lambda) + i\Delta{{f_m}''}(\lambda)]
    \exp(2\pi{i}(\mathbf{r}_m\cdot\mathbf{h}))
    \text{,}
    \label{eqn:vecF}
  \end{equation}

  Utilizing Euler's formula, $\exp(i\theta)=\cos(\theta)+i\sin(\theta)$ and expanding,
  
     \begin{equation}
    \mathbf{F}_{\mathbf{h},m} = [f^0_m + \Delta{f_m}\prime(\lambda) + i\Delta{{f_m}''}(\lambda)]
    [\cos(2\pi(\mathbf{r}_m\cdot\mathbf{h})) + i \sin(2\pi(\mathbf{r}_m\cdot\mathbf{h}))
    ]
    \text{,}
    \label{eqn:vecF}
  \end{equation}
  
      \begin{equation}
    \begin{array}{c}\mathbf{F}_{\mathbf{h},m} =  (f^0_m + \Delta{f_m}\prime(\lambda)) \cos(2\pi(\mathbf{r}_m\cdot\mathbf{h})) - 
    \Delta{{f_m}''}(\lambda)\sin(2\pi(\mathbf{r}_m\cdot\mathbf{h})) + \\
    i[ (f^0_m + \Delta{f_m}\prime(\lambda))\sin(2\pi(\mathbf{r}_m\cdot\mathbf{h})) +\Delta{{f_m}''}(\lambda)\cos(2\pi(\mathbf{r}_m\cdot\mathbf{h})) ]

    \end{array}
    \text{.}
    \label{eqn:vec2F}
  \end{equation}
  
  Switching to vector notation and rearranging,
   \begin{equation}
    \begin{array}{c}\mathbf{F}_{\mathbf{h},m} =  f^0_m [(1 + \dfrac{\Delta{f_m}\prime(\lambda)}{f^0_m}) \cos(2\pi(\mathbf{r}_m\cdot\mathbf{h})) - 
    \dfrac{\Delta{{f_m}''}(\lambda)}{f^0_m}\sin(2\pi(\mathbf{r}_m\cdot\mathbf{h})) ]+ \\
    if^0_m[ (1 + \dfrac{\Delta{f_m}\prime(\lambda)}{f^0_m})\sin(2\pi(\mathbf{r}_m\cdot\mathbf{h})) +
    \dfrac{\Delta{{f_m}''}(\lambda)}{f^0_m}\cos(2\pi(\mathbf{r}_m\cdot\mathbf{h})) ]

    \end{array}
    \text{,}
    \label{eqn:vecvecF}
  \end{equation}
    \begin{equation}
    \mathbf{F}_{\mathbf{h},m} =  f^0_m 
    \left(\begin{array} {c c} 
    1 + \dfrac{\Delta{f_m}\prime(\lambda)}{f^0_m}  & 
    \dfrac{-\Delta{{f_m}''}(\lambda)}{f^0_m} \\
     \dfrac{\Delta{{f_m}''}(\lambda)}{f^0_m} & 1 + \dfrac{\Delta{f_m}\prime(\lambda)}{f^0_m}
    \end{array}\right)
    \left(
    \begin{array}{c}
    \cos(2\pi(\mathbf{r}_m\cdot\mathbf{h})) \\
    \sin(2\pi(\mathbf{r}_m\cdot\mathbf{h}))
    \end{array}\right)
    \text{.}
    \label{eqn:matvecF}
  \end{equation}
    
  The purpose of expressing $\mathbf{F}_{\mathbf{h},m}$ in this form is so that there is a compact 
  notation for calculating derivatives for parameter optimization, where the relevant unknown parameters to
  determine are $\Delta{f_m}\prime(\lambda)$ and $\Delta{{f_m}''}$.  Specifically we have 
  
  \begin{equation}
   \dfrac{\partial\mathbf{F}_{\mathbf{h},m}} {\partial\Delta{f_m}\prime(\lambda)} = 
    \left(\begin{array} {c c} 
    1   & 0 \\
     0& 1 
    \end{array}\right)
    \left(
    \begin{array}{c}
    \cos(2\pi(\mathbf{r}_m\cdot\mathbf{h})) \\
    \sin(2\pi(\mathbf{r}_m\cdot\mathbf{h}))
    \end{array}\right)
    \text{,}
     \label{eqn:qthx}
  \end{equation}
and
\begin{equation}
   \dfrac{\partial\mathbf{F}_{\mathbf{h},m}} {\partial\Delta{{f_m}''}(\lambda)} =  
    \left(\begin{array} {c c} 
    0  & 
   -1 \\
     1 & 0
    \end{array}\right)
    \left(
    \begin{array}{c}
    \cos(2\pi(\mathbf{r}_m\cdot\mathbf{h})) \\
    \sin(2\pi(\mathbf{r}_m\cdot\mathbf{h}))
    \end{array}\right)
    \text{.}
    \label{eqn:qthy}
  \end{equation}

 Further modifications in the future to handle B-factors, ADPs, and occupancies.
  
 \subsection*{3 Derivatives of the intensity}
 
The intensity of a Miller index is 
     \begin{equation}
    I_{\mathbf{h}} = \mathbf{F}_{\mathbf{h}} \cdot \mathbf{F}_{\mathbf{h}}
    \text{.}
    \label{eqn:iall}
  \end{equation}
Suppose we wish to find the derivative of the model intensity with respect to some parameter $q$, for example
the $\Delta{f_m}\prime(\lambda)$ of a particular metal $m$.  Then
     \begin{equation}
    \frac{\partial{I}_{\mathbf{h}}}{\partial{q}} = 2\mathbf{F}_{\mathbf{h}} \cdot 
    \frac{\partial\mathbf{F}_{\mathbf{h}}} {\partial{q}}
    \text{.}
    \label{eqn:ialld}
  \end{equation}
Repeating the formula for structure factor:  it is a linear sum over many components in the unit cell:
       \begin{equation}
    \mathbf{F_h} =\left( \begin{array}{c}
      A \\
      B
    \end{array}\right)_{\textrm{bulk solvent}} +
    \sum_{\textrm{protein atoms}, p}\left( \begin{array}{c}
      a_p \\
      b_p
    \end{array}\right)+ 
    \sum_{m=1}^{N_m} \mathbf{F}_{\mathbf{h},m}+\textrm{other metals, etc.}
    \text{}
    \label{eqn:partsrecap}
  \end{equation}
where $N_m$ is the number of metals of that particular class, and  $\mathbf{F}_{\mathbf{h},m}$ is given by eqn. 
\eqref{eqn:matvecF}.
(Further modifications to add symmetry operators,
but for now assume we are in space group $P$1.)
Due to the linear nature of the structure factor summation the derivative passes through the sum, and the 
only terms to remain are those that specifically contain metals in that class:
       \begin{equation}
  \frac{\partial\mathbf{F}_{\mathbf{h}}} {\partial{q}} =   \sum_{m=1}^{N_m} 
  \frac {\partial\mathbf{F}_{\mathbf{h},m}}{\partial{q}}
      \text{.}
    \label{eqn:derOnq}
  \end{equation}

%%%%%%%%%%%%%%%%%%%%%
\end{document}
